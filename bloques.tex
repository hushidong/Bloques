\documentclass[a4paper,onecolumn]{IEEETran}
\usepackage{bloques}
\usepackage{verbatim}
\usepackage{enumerate}
\oddsidemargin=-1.5cm


\title{The Bloques Package}
\author{Alejandro Garces Ruiz \\ alejandrogarces@gmail.com}

\begin{document}

\maketitle

\section{Functions}


The \textbf{bloques} package is a very simple set of commands based on \textbf{tikz} to generate control blocks.  The only packages required in the definition are:

\begin{verbatim}
	\usepackage{tikz}
	\usepackage{bloques}
\end{verbatim}

The package is very efficient for sequential blocks as follow:

\begin{itemize}
\item \verb|\bStart{TEXT}| a start node without box
\item \verb|\bGain[mark]{TEXT}| a gain with box and a input mark before it
\item \verb|\bEnd{TEXT}| a start node without box
\item \verb|\bGainPlus{TEXT1}{TEXT2}| a gain(TEXT2) and plus mixer with input (TEXT1)
\item \verb|\bGainMinus{TEXT1}{TEXT2}| a gain(TEXT2) and minus mixer with input (TEXT1)
\item \verb|\bPlus[mark]{NODENAME}| a L-U plus mixer with name (NODENAME) and a input mark before it for feed
\item \verb|\bMinus[mark]{NODENAME}| a L-U minus mixer with name (NODENAME) and a input mark before it for feed
\item \verb|\bPlusLR[mark]{NODENAME}| a L-R plus mixer with name (NODENAME) and a input mark before it for feed
\item \verb|\bMinusLR[mark]{NODENAME}| a L-R minus mixer with name (NODENAME) and a input mark before it for feed
\item \verb|\bPlusF[mark]{NODENAME}| a L-D plus mixer with name (NODENAME)  for feed backward and a input mark before it
\item \verb|\bMinusF[mark]{NODENAME}| a L-D minus mixer with name (NODENAME)  for feed backward and a input mark before it
\item \verb|\bPlusDown{TEXT}| a plus mixer with a down input(TEXT)
\item \verb|\bPlusUp{TEXT}|   a plus mixer with a up input(TEXT)
\item \verb|\bMinusDown{TEXT}| a minus mixer with a down input(TEXT)
\item \verb|\bMinusUp{TEXT}| a minus mixer with a up input(TEXT)
\end{itemize}

IF want to start a new sequential blocks use these command:

\begin{itemize}
\item \verb|\bNewStart{TEXT}{POSITION}| a new start node with text(TEXT) at (POSITION)
\item \verb|\bMarkNode{NODENAME}| add a mark node with name {NODENAME} for the previous node
\item \verb|\bMarkNodeUp{NODENAME}| add a mark node with name {NODENAME} above ydistance of the previous node
\item \verb|\bMarkNodeDown{NODENAME}| add a mark node with name {NODENAME} below ydistance of the previous node
\item \verb|\bInter[mark]{TEXT}| a no sep-space inter node with text (TEXT1) and a previous (mark) for feed forward
\item \verb|\bMarkNodeInter{NODENAME}| a no sep-space inter node with name (NODENAME)  for feed forward
\item \verb|\bNewInter{TEXT}{POSITION}| a new inter node with text(TEXT) at (POSITION) like new start node but with no sep-space
\item \verb|\bFeedForward{TEXT}{NODE1}{NODE2}| a feed forward with gain (TEXT) from node (NODE1) to node (NODE2)
\item \verb|\bCrossGain{TEXT}{NODENAME1}{NODENAME2}| a cross gain with gain(TEXT) from node (NODENAME1) to node (NODENAME2)
\item \verb|\bLink{NODENAME1}{NODENAME2}| a Line link from node (NODENAME1) to node (NODENAME2)
\item \verb|\bLinkhv{NODENAME1}{NODENAME2}| a H-V Line link from node (NODENAME1) to node (NODENAME2)
\item \verb|\bLinkvh{NODENAME1}{NODENAME2}| a V-H Line link from node (NODENAME1) to node (NODENAME2)
\end{itemize}



For Feedback controls, it is required to mark the nodes with the following functions:

\begin{itemize}
\item \verb|\bFeedBack{TEXT}{NODENAME}| a feed backward with gain(TEXT) at below left to a mixer with name (NODENAME)
\item \verb|\bFeedBackA{TEXT}{NODENAME}| a feed backward with gain(TEXT) at above left to a mixer with name (NODENAME)
\item \verb|\bFeedBackvhv{TEXT}{NODENAME}| a feed backward with gain(TEXT) at below left to a mixer with name (NODENAME) and link line start from the south anchor of the previous node
\item \verb|\bFeedBackAvhv{TEXT}{NODENAME}| a feed backward with gain(TEXT) at above left to a mixer with name (NODENAME) and link line start from the north anchor of the previous node
\end{itemize}


To change colors and distances, the following functions are available

\begin{verbatim}
    \bShadow{NUMBER}  	% default = 0,shadow of node
    \bColorB{COLOR}   	% default = white, back color of node
    \bColorT{COLOR}		% default = black, text color of node
    \bLineL{Linestyle}  % default is none, more styles like dashed,double can be set
    \bArrow{Arrowstyle} % default is latex, more styles like stealth,Latex,Stealth can be set
    \ydistance{Length}	% default = 1.2 cm, offset distance of y direction
    \restoreydis    % recover to default of the offset distance of y direction
    \xdistance{Length}	% default = 1.5 cm, offset distance of x direction
    \restorexdis 	% recover to default of the offset distance of x direction
\end{verbatim}


\newpage

\section{examples}


\begin{figure}[!htbp]
\scalebox{0.9}{\begin{tikzpicture}
        %\bColorB{blue!70!green!20}
        \bLineL{double}
        \bArrow{-Latex[open]}
        \xdistance{1.4cm}
        \bStart{$v$}
        \bMinusF{Nodev}
        \bInter{}
        \bMarkNodeInter{Nodevf}
        \bGain[$u$]{$B$}
        \bPlusF{Nodeu}
        \bGain[$\dot{x}$]{$\int$}
        \bFeedBack{$A$}{Nodeu}
        \bGain[$x$]{$C$}
        \bInter{}
        \bInter[$y$]{}
        \bMarkNodeInter{Nodey}
        \bEnd{}
        %new branch
        \bNewInter{}{(2.8cm,-2.4cm)}
        \bMarkNodeInter{Nodeun}
        \xdistance{1.2cm}
        \bGain{$B$}
        \bMinus{Nodeuh}
        \bPlusF{Nodexh}
        \bGain[$\dot{\hat{x}}$]{$\int$}
        \xdistance{1.0cm}
        \bInter{}
        \bFeedBackvhv{$A$}{Nodexh}
        \ydistance{2.4cm}
        \bFeedBackvhv{$K$}{Nodev}
        \restoreydis
        \bGain[$\hat{x}$]{$C$}
        \bMinusLR[$\hat{y}$]{Nodeyh}
        \bFeedBackAvhv{$G$}{Nodeuh}
        \bLinkvh{Nodey}{Nodeyh}
        \bLinkhv{Nodevf}{Nodeun}
    \end{tikzpicture}}
    \caption{a feed back system with full dimension observer}
\end{figure}

%a state observer of a system
\begin{figure}[!htbp]
\scalebox{1.0}{\begin{tikzpicture}
        \bLineL{double}
        \bArrow{-Latex[open]}
        \bColorB{blue!70!green!20}
        \bStart{$u$}
        \bInter{}
        \bMarkNodeInter{NODEU}
        \bGain{$B$}
        \bPlusF{NODEX}
        \bGain[$\dot{x}$]{$\int$}
        \bFeedBack{A}{NODEX}
        \bGain[$x$]{$C$}
        \bInter{}
        \bMarkNodeInter{NODEy}
        \bEnd{$y$}
        %observer
        \bNewInter{}{(\xdistancia,-2)}
        \bMarkNodeInter{NODEua}
        \bGain{$B$}
        \xdistance{1cm}
        \bMinusF{NODEXa}
        \bPlusF{NODEXb}
        \bGain[$\dot{\hat{x}}$]{$\int$}
        \bFeedBack{A}{NODEXb}
        \restorexdis
        \bGain[$\hat{x}$]{$C$}
        \bMinus{NODEyb}
        \ydistance{2.5cm}
        \bFeedBackvhv{G}{NODEXa}
        \bLink{NODEU}{NODEua}
        \bLinkhv{NODEy}{NODEyb}
    \end{tikzpicture}}
    \caption{a state observer of a system}
\end{figure}



%add example 2021-12-28
%A system state variables diagram
\begin{figure}[h!]
\hspace{-0.5cm}
	\begin{tikzpicture}
        \bColorB{blue!70!green!20}
        \bStart{$u$}
        \bInter{}
        \bMarkNodeInter{NODEU}
        \bGain{$B$}
        \bPlusF{NODEX}
        \bGain[$\dot{x}$]{$\int$}
        \bFeedBack{A}{NODEX}
        \bGain[$x$]{$C$}
        \bPlus{NODEY}
        \bFeedForward{$D$}{NODEU}{NODEY}
        \bEnd{$y$}
		\node[right of = NODO1, text width=8cm, node distance = 4.3cm, drop shadow={opacity=1}, fill=blue!20, rounded corners] (C4)
	{
	\begin{verbatim}
    \begin{figure}
    	\begin{tikzpicture}
            \bColorB{blue!70!green!20}
            \bStart{$u$}
            \bInter{}
            \bMarkNodeInter{NODEU}
            \bGain{$B$}
            \bMinusF{NODEX}
            \bGain[$\dot{x}$]{$\int$}
            \bFeedBack{A}{NODEX}
            \bGain[$x$]{$C$}
            \bPlus{NODEY}
            \bFeedForward{$D$}{NODEU}{NODEY}
            \bEnd{$y$}
    	\end{tikzpicture}
    \end{figure}
	\end{verbatim}
	};
	\end{tikzpicture}
	\caption{A system state variables diagram}
\end{figure}


%Simple Control diagram
\begin{figure}[h!]
\begin{tikzpicture}	
	\bStart{$X_{ref}$}
	\bMinusDown{$X$}
	\bGain{$k_{p}$}	
	\bPlusUp{$Y$}
	\bGainPlus{$U$}{$\omega L$}
	\bEnd{$V_{x}$}
	\node[right of = NODO1, text width=8.5cm, node distance = 6.0cm, drop shadow={opacity=1}, fill=blue!20, rounded corners] (C4)
	{
	\begin{verbatim}	
	\begin{figure}
	\begin{tikzpicture}	
	   \bStart{$X_{ref}$}
	   \bMinusDown{$X$}
	   \bGain{$k_{p}$}	
	   \bPlusUp{$Y$}
	   \bGainPlus{$U$}{$\omega L$}
	   \bEnd{$V_{x}$}	
	\end{tikzpicture}
	\end{figure}
	\end{verbatim}
	};
\end{tikzpicture}
\caption{Simple Control diagram}
\end{figure}



%Control diagram with shadow
\begin{figure}[h!]
\begin{tikzpicture}	
	\bShadow
	\bStart{$X_{ref}$}
	\bMinusDown{$X$}
	\bGain{$k_{p}$}	
	\bPlusUp{$Y$}
	\bGainPlus{$U$}{$\omega L$}
	\bEnd{$V_{x}$}
	\node[right of = NODO1, text width=8.5cm, node distance = 6.0cm, drop shadow={opacity=1}, fill=blue!20, rounded corners] (C4)
	{
	\begin{verbatim}	
	\begin{figure}
	\begin{tikzpicture}	
	 \bShadow
	    \bStart{$X_{ref}$}
	    \bMinusDown{$X$}
	    \bGain{$k_{p}$}	
	    \bPlusUp{$Y$}
	    \bGainPlus{$U$}{$\omega L$}
	    \bEnd{$V_{x}$}	
	\end{tikzpicture}
	\end{figure}
	\end{verbatim}
	};
\end{tikzpicture}
\caption{Control diagram with shadow}
\end{figure}



%Control diagram with shadow and different colors
\begin{figure}[h!]
\begin{tikzpicture}[thick]	
 \draw[fill=blue!20, draw=white]  (-0.5,-3) rectangle (8,2);
 \draw[fill=green!20, dashed,thick] (4,-2) rectangle (7,0.5);
  \bShadow
  \bColorB{blue!80!green!50}
	\bColorT{yellow}
	\bColorL{white}
	\bStart{$X_{ref}$}
	\bMinusDown{$X$}
	\bGain{$k_{p}$}	
	\bColorB{blue!30!green!80}
	\bPlusUp{$Y$}
	\bGainPlus{$U$}{$\omega L$}
	\bEnd{$V_{x}$}
	\node[right of = NODO1, text width=8.5cm, node distance = 6.0cm, drop shadow={opacity=1}, fill=blue!20, rounded corners] (C4)
	{
	\begin{verbatim}	
	\begin{figure}
	\begin{tikzpicture}[thick]
			\draw[fill=blue!20, draw=white]
			     (-0.5,-3) rectangle (8,2);
	    \draw[fill=green!20, dashed]
	         (4,-2) rectangle (7,0.5);
	 \bShadow
	 \bColorB{blue!50!green!45}
	 \bColorT{yellow}
	 \bColorL{white}
	    \bStart{$X_{ref}$}
	    \bMinusDown{$X$}
	    \bGain{$k_{p}$}	
	 \bColorB{blue!30!green!80}
	    \bPlusUp{$Y$}
	    \bGainPlus{$U$}{$\omega L$}
	    \bEnd{$V_{x}$}	
	\end{tikzpicture}
	\end{figure}
	\end{verbatim}
	};
\end{tikzpicture}
\caption{Control diagram with shadow and different colors}
\end{figure}



%Control diagram with feedback
\begin{figure}[h!]
	\begin{tikzpicture}
		\bColorB{blue!70!green!20}		
		\bStart{$X_{r}$}
		\bMinusF{NODEX}
		\bGain{$k_{p}+\frac{k_{i}}{S}$}
		\bFeedBack{$k_{x}$}{NODEX}
		\bEnd{$X$}
	\node[right of = NODO1, text width=8.5cm, node distance = 6.0cm, drop shadow={opacity=1}, fill=blue!20, rounded corners] (C4)
	{
	\begin{verbatim}	
        \begin{figure}
        	\begin{tikzpicture}
        		   \bColorB{blue!70!green!20}		
        		   \bStart{$X_{r}$}
        		\bMinusF{NODEX}
        		   \bGain{$k_{p}+\frac{k_{i}}{S}$}
        		\bFeedBack{$k_{x}$}{NODEX}
        		   \bEnd{$X$}		
        	\end{tikzpicture}
        \end{figure}	
	\end{verbatim}		
	}	;
	\end{tikzpicture}
	\caption{Control diagram with feedback}
\end{figure}



%Change the ydistance
\begin{figure}[h!]
	\begin{tikzpicture}
		\bColorB{blue!30!green!50}
		\bColorT{white}
		\bStart{$X_{r}$}
		\bMinusF{NODEX}
		\bGain{$k_{p}+\frac{k_{i}}{S}$}
		\bGain{$K_{2}$}
		\bPlusDown{$Y$}
		\ydistance{2.5cm}				
		\bFeedBack{$k_{x}$}{NODEX}
		\bEnd{$X$}
\node[right of = NODO1, text width=8.5cm, node distance = 6.0cm, drop shadow={opacity=1}, fill=blue!20, rounded corners] (C4)
	{
	\begin{verbatim}	
        \begin{figure}
        	\begin{tikzpicture}
        	    \bColorB{blue!30!green!50}
        	    \bColorT{white}
        	    \bStart{$X_{r}$}
        	    \bMinusF{NODEX}
        	    \bGain{$k_{p}+\frac{k_{i}}{S}$}
        	    \bGain{$K_{2}$}
        	    \bPlusDown{$Y$}
        		\ydistance{2.5cm}				
        	    \bFeedBack{$k_{x}$}{NODEX}
        	    \bEnd{$X$}
        	\end{tikzpicture}
        \end{figure}
	\end{verbatim}		
	}	;		
	\end{tikzpicture}
	\caption{Change the ydistance}
\end{figure}


%More compex controls
\begin{figure}[h!]
	\begin{tikzpicture}
		\bStart{$I_{d(ref)}=0$}
		\bMinusDown{$I_{d}$}
		\bGain{PI1}
		\bPlusF{NODET}
		\bEnd{$V_{d}$}
		\bNewStart{$\omega_{ref}$}{(-2,-4)}
		\bMinusDown{$\omega_{s}$}
		\bGain{PI2}
		\bMinusUp{$I_{q}$}
		\bMarkNodeUp{NODEX}
		\bGain{PI3}
		\bEnd{$V_{q}$}		
		\bCrossGain{$\omega L$} {NODEX} {NODET}
		\node[right of = NODO1, text width=8.5cm, node distance = 6.0cm, drop shadow={opacity=1}, fill=blue!20, rounded corners] (C4)
	{
	\begin{verbatim}	
        \begin{figure}
        	\begin{tikzpicture}
        		\bStart{$I_{d(ref)}=0$}
        		   \bMinusDown{$I_{d}$}
        		   \bGain{PI1}
        		   \bPlusF{NODET}
        		   \bEnd{$V_{d}$}
        		\bNewStart{$\omega_{ref}$}{(-2,-4)}
        		   \bMinusDown{$\omega_{s}$}
        		   \bGain{PI2}
        		   \bMinusUp{$I_{q}$}
        		\bMarkNodeUp{NODEX}
        		   \bGain{PI3}
        		   \bEnd{$V_{q}$}		
        		\bCrossGain{$\omega L$} {NODEX} {NODET}
        	\end{tikzpicture}
        \end{figure}		
	\end{verbatim}
	};
	\end{tikzpicture}
	\caption{More compex controls}
\end{figure}






\clearpage

\section{history}

\begin{itemize}
  \item  update on 2021-12-28, by hu zhenzhen (hzzmail@163.com)
    \begin{itemize}
    \item add new some commands for feed forward drawing
    \item add more instructions of cmds in the doc
    \end{itemize}

  \item v1.0 in 2005, uses TikZ to provide commands for generating control diagrams (specially in power electronics)
\end{itemize}


\end{document}




